% !TEX TS-program = pdflatex
% !TEX encoding = UTF-8 Unicode

% This is a simple template for a LaTeX document using the "article" class.
% See "book", "report", "letter" for other types of document.

\documentclass[11pt]{article} % use larger type; default would be 10pt

\usepackage[utf8]{inputenc} % set input encoding (not needed with XeLaTeX)
\usepackage{tikz}
%%% Examples of Article customizations
% These packages are optional, depending whether you want the features they provide.
% See the LaTeX Companion or other references for full information.

%%% PAGE DIMENSIONS
\usepackage{geometry} % to change the page dimensions
\geometry{a4paper} % or letterpaper (US) or a5paper or....
% \geometry{margins=2in} % for example, change the margins to 2 inches all round
% \geometry{landscape} % set up the page for landscape
%   read geometry.pdf for detailed page layout information

\usepackage{graphicx} % support the \includegraphics command and options

% \usepackage[parfill]{parskip} % Activate to begin paragraphs with an empty line rather than an indent

%%% PACKAGES
\usepackage{booktabs} % for much better looking tables
\usepackage{array} % for better arrays (eg matrices) in maths
\usepackage{paralist} % very flexible & customisable lists (eg. enumerate/itemize, etc.)
\usepackage{verbatim} % adds environment for commenting out blocks of text & for better verbatim
\usepackage{subfig} % make it possible to include more than one captioned figure/table in a single float
% These packages are all incorporated in the memoir class to one degree or another...
\usepackage{url}
\usepackage{hyperref}

%%% HEADERS & FOOTERS
\usepackage{fancyhdr} % This should be set AFTER setting up the page geometry
\pagestyle{fancy} % options: empty , plain , fancy
\renewcommand{\headrulewidth}{0pt} % customise the layout...
\lhead{}\chead{}\rhead{}
\lfoot{}\cfoot{\thepage}\rfoot{}

%%% SECTION TITLE APPEARANCE
\usepackage{sectsty}
\allsectionsfont{\sffamily\mdseries\upshape} % (See the fntguide.pdf for font help)
% (This matches ConTeXt defaults)

%%% ToC (table of contents) APPEARANCE
\usepackage[nottoc,notlof,notlot]{tocbibind} % Put the bibliography in the ToC
\usepackage[titles,subfigure]{tocloft} % Alter the style of the Table of Contents
\renewcommand{\cftsecfont}{\rmfamily\mdseries\upshape}
\renewcommand{\cftsecpagefont}{\rmfamily\mdseries\upshape} % No bold!

%%% END Article customizations

%%% The "real" document content comes below...

\usepackage{amsfonts}
\usepackage{amsmath}
\usepackage{amsthm}

\theoremstyle{definition}
\newtheorem*{beispiel}{Beispiel}
\newtheorem{definition}{Definition}
\newtheorem*{bemerkung}{Bemerkung}
\newtheorem*{beweis}{Beweis}
\newtheorem*{ubung}{Übung}
\newtheorem{theorem}{Theorem}

\title{Wissenschaftlicher, rechnender Schreibomat}
\author{Florian}
%\date{} % Activate to display a given date or no date (if empty),
         % otherwise the current date is printed 

\begin{document}
\maketitle

\section*{Tools}

\section{Huffmann-Codierung}

führt immer zu einem regulären binären Baum.

\begin{definition}[Entropie]
Durchschnittliche Anzahl Zeichen, die für Zeichen des Alphabets $A$ verwendet werden:
\[
E(A) = \sum\limits_{x \in A} p(x) \cdot \log_2 \frac{1}{p(x)}
\]
\end{definition}

\section{Hamming Code}

(4,7)-Code: 4 Informationsbits, 3 Parity-Bits.

Nachricht, die in den Kanal verschickt werden soll:

$(D_7, D_6, D_5, D_3)$

Führt zu:

$(D_7, D_6, D_5, P_4, D_3, P_2, P_1)$

\[
\begin{bmatrix}
1 & 1 & 1 & \vline & 1 \\
1 & 0 & 1 & \vline & 0 \\
0 & 0 & 0 & \vline & 0
\end{bmatrix}
\]

P4 = D7 + D6 + D5,

P2 = D7 + D6 + D3,

P1 = D7 + D5 + D3

Generator-Matrix $G$, Checkmatrix $H$:

\begin{eqnarray*}
G &=& \begin{bmatrix}
	D_7 & D_6 & D_5 & P_4 & D_3 & P_2 & P_1
\end{bmatrix} \\
&=& \begin{bmatrix}
1 & 0 & 0 & 1 & 0 & 1 & 1 \\
0 & 1 & 0 & 1 & 0 & 1 & 0\\
0 & 0 & 1 & 1 & 0 & 0 & 1\\
0 & 0 & 0 & 0 & 1 & 1 & 1
\end{bmatrix} \begin{bmatrix}
 D_7 & D_6 & D_5 & D_3
\end{bmatrix} \\
H &=& \begin{bmatrix}
D_7 & D_6 & D_5 & P_4 & D_3 & P_2 & P_1
\end{bmatrix} \\
&=& \begin{bmatrix}
1 & 1 & 1 & 1 & 0 & 0 & 0\\
1 & 1 & 0 & 0 & 1 & 1 & 0\\
1 & 0 & 1 & 0 & 1 & 0 & 1
\end{bmatrix} \begin{bmatrix} P_4 & P_2 & P_1 \end{bmatrix}
\end{eqnarray*}

\section{Konditionierung einer Matrix}

\begin{beispiel} Finden einer Lösung. $y$ variiert: was bedeutet das für $x$?
\begin{eqnarray*}
2x + y &=& 3 \\
2x + (1.001)y &=& 0 \\
\Rightarrow y &=& 3 - 2x \\
\end{eqnarray*}
\end{beispiel}

\begin{beispiel} Ist das Gleichungssystem singulär? $\det (A) = 0$
\begin{eqnarray*}
A &=& \begin{bmatrix}
2.1 & -0.6 & 1.1 \\
3.2 & 4.7 & -0.8 \\
3.1 & -6.5 & 4.1
\end{bmatrix} \\
\Rightarrow \det (A) &=& 2.1\cdot 4.7 \cdot 4.1 + (-0.6)\cdot (-0.8) \cdot 3.1 + 1.1\cdot 3.2 \cdot (-6.5)
\\ && -3.1\cdot 4.7 \cdot 1.1 - (-6.5)\cdot (-0.8) \cdot 2.1 - 4.1 \cdot 3.2 \cdot -0.6 \\
&=& 0
\end{eqnarray*}
\end{beispiel}

\section{Das Gauss'sche Eliminationsverfahren}

\begin{beispiel} Nun.

\[
\begin{bmatrix}
4 & -2 & 1 &\vline& 11 \\
-2 & 4 & -2 &\vline& -16 \\
1 & -2 & 4 &\vline& 17
\end{bmatrix}
\]
Nach $(b) - (-0.5)\cdot (a)$:

\[
\begin{bmatrix}
4 & -2 & 1 &\vline& 11 \\
0 & 3 & -1.5 &\vline& -10.5 \\
1 & -2 & 4 &\vline& 17
\end{bmatrix}
\]

Nach $(c) - (-0.25)\cdot (a)$:

\[
\begin{bmatrix}
4 & -2 & 1 &\vline& 11 \\
0 & 3 & -1.5 &\vline& -10.5 \\
0 & -1.5 & 3.75 &\vline&14.25
\end{bmatrix}
\]

Nach $(c) - (-0.5)\cdot (b)$:

\[
\begin{bmatrix}
4 & -2 & 1 &\vline& 11 \\
0 & 3 & -1.5 &\vline& -10.5 \\
0 & 0 & 3 &\vline&9
\end{bmatrix}
\]

\[
\Rightarrow \begin{bmatrix}
4 & -2 & 1  \\
0 & 3 & -1.5  \\
0 & 0 & 3 
\end{bmatrix} \begin{bmatrix}
x_1 \\ x_2 \\ x_3
\end{bmatrix} = \begin{bmatrix}
 11 \\
 -10.5 \\
 9
\end{bmatrix}
\]

\end{beispiel}

\begin{beispiel}
\[
\begin{bmatrix}
-6 & -4 & 1 &\vline& 22 \\ 
-4 & 6 & -4 &\vline& -18 \\ 
1 & -4 & 6 &\vline& 11
\end{bmatrix}
\Rightarrow
\begin{bmatrix}
-6 & -4 & 1 &\vline& 22 \\ 
0 & \frac{26} 3 & -\frac{14} 3 &\vline& -\frac{10} 3 \\ 
0 & -\frac{10} 3 & 6 &\vline& 11
\end{bmatrix}
\]
\end{beispiel}

\section{LU-Dekomposition}

\begin{theorem}
Jede symmetrische Matrix kann sich in eine LU-Dekomposition zerlegen lassen.
\end{theorem}

\begin{description}

\item[Dekomposition von Doolittle]

\[
L_{ii} = 1
\]

\item[Dekomposition von Count]

\[
U_{ii} = 1
\]

\item[Dekomposition von Cholesky]

\[
L = U^T
\]

Die Nebenbedingung ist, dass die Matrix $A$ positiv definit ist:

\[
\vec x^T A \vec x > 0
\]

\end{description}

Warum LU-Dekomposition? Gleichungssystem: $Ax = b, A = L\cdot U \Rightarrow LUx = b$. Finde Lösung für $Ly = b$, dann $Ux = y$

\subsection{Dekomposition von Doolittle}

\[
L = \begin{bmatrix}
1 & 0 & 0 \\
L_{21} & 1 & 0 \\
L_{31} & L_{32} & 1 \\
\end{bmatrix}, U = \begin{bmatrix}
U_{11} & U_{12} & U_{13} \\
0 & U_{22} & U_{23} \\
0 & 0 & U_{33}
\end{bmatrix}
\]
\[
\Rightarrow L\cdot U = \begin{bmatrix}
U_{11} & U_{12} & U_{13} \\
U_{11} L_{21} & U_{11} L_{21} + U_{22} & L_{21} U_{13} + U_{23} \\
U_{11} L_{31} & U_{12} L_{31} + U_{22} L_{32} & U_{13} L_{31} + U_{23} L_{32} + U_{33}
\end{bmatrix}
\]

Darauf wird nun das Gauss'sche Eliminationsverfahren angewandt:

2. Reihe: 2. Reihe - $L_{21}\cdot$ 1. Reihe
3. Reihe: 3. Reihe - $L_{31}\cdot$ 1. Reihe

\[
\Rightarrow A' = \begin{bmatrix}
U_{11} & U_{12}  & U_{13} \\
0 & U_{22} & U_{23} \\
0 & U_{22}L_{32} & U_{23}L_{32} + U_{33}
\end{bmatrix}
\]

Neuer Gauss'scher Filterschritt:

3. Reihe: 3. Reihe - $L_{32}\cdot$ 2. Reihe

\[
\Rightarrow A'' = \begin{bmatrix}
U_{11} & U_{12}  & U_{13} \\
0 & U_{22} & U_{23} \\
0 & 0 & U_{33}
\end{bmatrix}
\]

\subsection{Dekomposition von Cholesky}

\[
A= LL^T
\]

\begin{eqnarray*}
A &=& \begin{bmatrix}
A_{11} & A_{21} & A_{31} \\
A_{12} & A_{22} & A_{32} \\
A_{13} & A_{23} & A_{33}
\end{bmatrix} = \begin{bmatrix}
L_{11} & 0 & 0 \\
L_{21} & L_{22} &  0 \\
L_{31} & L_{32} & L_{33}
\end{bmatrix} \cdot \begin{bmatrix}
L_{11} & L_{21} & L_{31} \\
0 & L_{22} & L_{32} \\
0 & 0 & L_{33}
\end{bmatrix} \\
&=& \begin{bmatrix}
L_{11}^2 & L_{11}L_{21} & L_{11}L_{31} \\
L_{21}L_{11} & L_{21}^2 + L_{22}^2 & L_{21}L_{31} + L_{22}L_{32} \\
L_{31}L_{11} & L_{31}L_{32} + L_{32}L_{22} & L_{31}^2 + L_{32}^2 + L_{33}^2
\end{bmatrix}
\end{eqnarray*}

Aha.

\begin{eqnarray*}
A_{11} = L_{11}^2 &\leftrightarrow& L_{11} = \sqrt{A_{11}} \\
A_{21} = L_{11}L_{21} &\leftrightarrow& L_{21} = \frac{A_{21}}{L_{11}} = \frac{A_{21}}{\sqrt{A_{11}}} \\
A_{31} = L_{11}L_{31} &\leftrightarrow& L_{31} = \frac{A_{31}}{L_{11}} = \frac{A_{31}}{\sqrt{A_{11}}} \\
&& L_{32} = \frac{(A_{32} - L_{21}L_{31})}{L_{22}} \\
L_{22} = \sqrt{A_{22} - L_{21}^2} && \\
L_{31} = \frac{A_{32} - L_{32}L_{22}}{L_{21}} && \\
A_{33} = L_{31}^2-L_{32}^2 + L_{32}^2 &\leftrightarrow& L_{33} = \sqrt{A_{33}- (L_{31}L_{32})}
\end{eqnarray*}

\begin{beispiel}
\begin{eqnarray*}
A &=& \begin{bmatrix}
4 & -2 & 2 \\ -2 & 2 & -4 \\ 2 & -4 & 11
\end{bmatrix} \\
L_{11} &=& \sqrt{A_{11}} = 2 \\
L_{21} &=& \frac{A_{21}}{2} = -1 \\
L_{31} &=& \frac{A_{31}}{L_{11}} = 1 \\
L_{22} &=& \sqrt{A_{22} - L_{21}^2} = 1 \\
L_{32} &=& (-4) + 1 = -3 \\
L_{33} &=& \sqrt{11 - (1+3^2)} = \sqrt{1} = 1
\end{eqnarray*}

Cholesky-Zerlegung gefunden:
\[
\begin{bmatrix}
2 & 0 & 0 \\
-1 & 1 & 0 \\
1 & -3 & 1
\end{bmatrix}
\begin{bmatrix}
x_1 \\ x_2 \\ x_3
\end{bmatrix} = \begin{bmatrix}
c_1 \\ c_2 \\ c_3
\end{bmatrix}
\] 

\end{beispiel}

\section{Positiv definit}

Matrix $A$ positiv definit äquivalent zu

\[
\begin{bmatrix} x_1 & x_2 & x_3 \end{bmatrix} A \begin{bmatrix} x_1 \\ x_2 \\ x_3 \end{bmatrix} \ge 0
\]

Zeigen dass das immer gilt!


\end{document}
